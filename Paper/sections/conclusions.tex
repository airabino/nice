\section{Conclusions}

In most senses, \glspl{bev} behave similarly or identically to \glspl{icev}. The primary advantage of \glspl{bev} in day-to-day use is the flexibility and economy afforded by charging compared to fueling. This strength becomes a drawback for long trips where slower energizing rates and less mature infrastructure impose a travel-time tax on \gls{bev} drivers substantially above that imposed on \gls{icev} drivers. Charging speeds and charging equipment costs are technical issues which may be fundamental. Much more readily addressable are the inefficient structure of the DC \gls{esn} and the policies which have guided its development.

Infrastructure networks often develop in three phases; connecting, balancing, and hardening. In the first phase minimum service is provided, in the second phase high demand elements are upgraded, and in the third phase, redundant elements are added to maintain functionality in the case of element saturation or failure. Optimal network expansion problems reduce to a simple question: how to best allocate the next batch of resources. In the connecting phase, any additional resource should be used to increase the portion of potential users that can utilize to the network. In turn, the metric of optimization is simply connectivity. In the balancing phase the metric of optimization becomes performance.

This paper introduces a novel methodology for the optimal expansion of \glspl{esn} sensitive to station congestion. This methodology represents a paradigm shift from a connectivity mindset ot a balancing mindset. This paper introduces the methodology and deploys it in a relevant and important real-life case study on a large network. The Case study, concerning the central California highway corridor DC \gls{esn} demonstrates the impact network structure on performance as demand increases. The case study also shows the utility of the methodology in generating cost-efficient improvements to existing \glspl{esn}. 