\section{Introduction}

%Infrastructure networks tend to evolve in three phases; connecting, balancing, and hardening. In the first phase minimum service is provided for all users. In the second phase high usage elements are upgraded to meet observed demand. In the third phase, redundant elements are added in order to accommodate overflow demand and to maintain functionality in the case of element failure. 

\gls{bev} charging infrastructure is relatively new when compared to other transportation infrastructure systems. Per \gls{afdc}, the number of public charging stations in the US increased by more than 800\% in the decade between 2013 and 2023 [SOURCE AFDC 10972]. The number is certainly much higher when factoring in private charging stations but less is known about these \cite{Davis_2022}. Most of these stations are small AC charging stations and serve primarily to meet daily travel needs. Of the stations currently tracked by \gls{afdc}, about 16.5\% are DC charging stations which must, by necessity, serve as the backbone of long distance \gls{bev} travel.

The current charging network in the US is spatially unequal. In some areas of the country the DC charging network is quite sparse and, in others, quite dense. There are only 66 total DC stations in the state of Idaho with a total of 193 ports between them. The vast majority of these are on Interstate highways between the large population centers of Boise and Idaho Falls. By contrast, California contains nearly 2,500 DC stations with over 14,000 ports. These networks are, plainly, in different phases of their development. 

Infrastructure networks tend to develop in three phases; connecting, balancing, and hardening. In the first phase minimum service is provided, in the second phase high demand elements are upgraded, and in the third phase, redundant elements are added to maintain functionality in the case of element saturation or failure. Optimal network expansion problems reduce to a simple question: how to best allocate the next batch of resources. In the connecting phase, any additional resource should be used to increase the portion of potential users that can utilize to the network. I turn, the metric of optimization is simply connectivity. Once full connectivity is achieved, the problem becomes more complex.

The purpose of a travel infrastructure network is to induce travel by reducing its cost. Different users will evaluate cost differently [SOURCE] and, oftentimes, the difference in costs between multiple routes will fall within an individual's threshold of disambiguation [SOURCE]. In general, travelers are sensitive to prices and travel times both of which are results of network design. Travel-mode specific characteristics matter in optimizing design. \glspl{bev} and \glspl{icev} share the same roads but draw energy from entirely separate networks. This means that for sufficiently long trips, they should be considered as different travel modes. At a typical US fueling rate of 7 gallons per minute [SOURCE], a gasoline powered vehicle is adding energy at a rate of 14.15 MW. High-end \glspl{bev} charge at maximum rates of around 350 kW. \glspl{bev} are 3-5 times more efficient [SOURCE] but the effect is that \gls{bev} charging events are substantially longer than \glspl{icev} fueling events. DC chargers are, also, more expensive to install than liquid fuel pumps [SOURCE] with the cost depending on a number of factors including the cost of upgrading the power grid to support the station \cite{Gamage_2023}. The cost differential meas that fueling station operators can add capacity to meet demand far easier than charging station operators can. As a result, fueling station congestion is not usually an issue but charging station congestion is an emerging adoption bottleneck [SOURCE].

The performance of an \gls{esn} (charging/fueling network) is the travel-time tax imposed on users by the structure of the network. The travel-time tax is the amount of travel-time added due to the need to charge/fuel. A better designed \gls{esn} imposes a smaller tax. \gls{esn} structure increases travel-time in three ways: 1) time required to detour from the shortest route to use a station, 2) time spent queuing at the station before the charging/fueling event, and 3) time spent during the charging/fueling event.

The literature on optimal charging/fueling network expansion is dominated by methodology well suited to fueling network design. In this paradigm stations are a fungible unit. The most significant time penalty is due to detouring. Thus, the decision of where to locate stations is the most important determinant of network performance. Once in place, a station may be designed to store and dispense fuel as needed. For charging networks, a different paradigm is needed. Because charging events are long, queues are likely to form. The number of chargers, power dispensing capability of the chargers, and physical layout of the station impact how efficiently the queue will clear, and thus the actual capacity of the station, in nonlinear but predictable ways. Adding a given number of stations or chargers can lead to very different performance gains depending on the manner in which they are allocated.

This paper presents a novel method for charging network optimal expansion accounting for all elements of travel-time tax. This method is an improvement over previous methodology for planning the expansion of charging networks, particularly those which are in the balancing phase of development. Specific contributions of this paper are:

\begin{enumerate}
	\item A novel formulation for the optimal \gls{esn} expansion problem accounting for expected queuing delay at stations. The optimization code is provided as a Python package.
	\item An application of said formulation to a real-life case study with generalizable takeaways.
\end{enumerate}


%In the connecting phase, it is sufficient to ask where should stations be located such that all origin-destination pairs are connected. Where the DC charging network has guaranteed connectivity, the next priority is balancing demand and supply. It is pertinent to note that the amount of "supply" a network provides is not captured by a single number such as total stations or total chargers. The composition of a network in terms of locations, speeds, and port counts of stations matter but cannot be evaluated independent of demand. Much research has been performed in the field of charging network capacity expansion optimization. The vast majority of this research is connectivity-oriented and ignores the effects of station composition.