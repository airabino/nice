\section{Introduction}

Infrastructure networks tend to evolve in three phases; connecting, balancing, and hardening. In the first phase minimum service is provided for all users. In the second phase high usage elements are upgraded to meet observed demand. In the third phase, redundant elements are added in order to accommodate overflow demand and to maintain functionality in the case of element failure. \gls{bev} charging infrastructure is relatively new when compared to other transportation infrastructure systems. Per \gls{afdc}, the number of public charging stations in the US increased by more than 800\% in the decade between 2013 and 2023 [SOURCE AFDC 10972]. The number is certainly much higher when factoring in private charging stations but less is known about these \cite{Davis_2022}. Most of these stations are small AC charging stations and serve primarily to meet daily travel needs. Of the stations currently tracked by \gls{afdc}, about 16.5\% are DC charging stations which must, by necessity, serve as the backbone of long distance \gls{bev} travel. 

The current charging network in the US is spatially unequal. In some areas of the country the DC charging network is quite sparse and, in others, quite dense. There are only 66 total DC stations in the state of Idaho with a total of 193 ports between them. The vast majority of these are on the interstate highway between the large population centers of Boise and Idaho Falls. By contrast, California contains nearly DC 2,500 stations and 14,000 ports. These networks are, plainly, in different phases of their development with different investment priorities. It is important to provide methodology which fits each phase.

Figure with cut sets for CA and ID

In the connecting phase, it is sufficient to ask where should stations be located such that all origin-destination pairs are connected. Where the DC charging network has guaranteed connectivity, the next priority is balancing demand and supply. It is pertinent to note that the amount of "supply" a network provides is not captured by a single number such as total stations or total chargers. The composition of a network in terms of locations, speeds, and port counts of stations matter but cannot be evaluated independent of demand. Much research has been performed in the field of charging network capacity expansion optimization. The vast majority of this research is connectivity-oriented and ignores the effects of station composition.