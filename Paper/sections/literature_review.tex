\section{Literature Review}

The the problem of optimal \gls{esn} expansion belongs to the larger field of strategic facility location optimization, a fundamental problem in operations research \citep{Owen_1998}. The optimal \gls{esn} expansion problem seeks to find the set of station locations and characteristics which most efficiently serves a set of demands. The literature on the optimal \gls{esn} expansion problem can be categorized by scope and approach.

\subsection{Scope}

\subsubsection{Demand}

Considered from a system-level perspective, the underlying goal of the optimization is multi-objective: to find a solution which minimizes user costs and infrastructure costs. The optimal \gls{esn} expansion problem must begin with a definition of demand, user costs, and infrastructure costs.

Demand may be defined by locations (nodal), or trips/tours (travel) \citep{Metais_2022_rser}. Nodal demand formulations attempt to minimize the cost of relocating from locations where users will be when they desire to charge to charging locations. This is accomplished by assigning equipment to locations and demand nodes to equipment. Nodal demand studies have employed various formulations to optimize around technical, economic, and environmental objectives and constraints \citep{Zhu_2018_jssse, Yi_2019_e, Xiao_2020_jes, Ma_Xie_2021_trd, Kuby_2023_he, Faustino_2023_e, Gupta_2023_jes, Liu_2023_ijst, Weekx_2024_trr, Yuvaraj_2024_ia, Davatgari_2024_ejor, Tungom_2024_esa, Vijay_2024_es, Wu_2024_trd}. Nodal demand problems are most relevant to planning on a municipal scale. Most travel is routine and short-distance \citep{NHTS_2017, NHTS_2022}, and most charging is accomplished during long-dwells \citep{Hardman_2018}. Access to low-rate charging is an important aspect of \gls{bev} user convenience \citep{Rabinowitz_2023_ia}.

Travel demand formulations attempt to minimize the added cost of charging/fueling for itineraries which are likely to exceed vehicle range. This is accomplished by assigning equipment to locations and flows to equipment. Charging and fueling add time to trips because of the time required to charge/fuel, the time required to wait for available equipment, and the time required to deviate from one's route to reach a station. Flow-capturing formulations optimize an \gls{esn} to best serve traffic flows along the routes they normally take, often the shortest path \citep{Kuby_2005_seps, Kuby_2007_nse, Upchurch_2009_ga}. Flow-enabling formulations optimize and \gls{esn} to best enable traffic flows by dictating what routes they will take from a set of possible paths \citep{Kim_2012_ijhe, MirHassani_2013_ts, Huang_Li_2015_nse, Li_Huang_2016_trc, Zhang_2017_trb, Tian_2018_s, Arslan_2019_ts, Anjos_2020_ejor, Wu_2024_e, Zeng_2024_mt, Ala_2024_scis, Pourvaziri_2024_tre}. Travel demand problems are most relevant to regional-scale planning.

\subsubsection{User Costs}

Maximizing level-of-service is accomplished by minimizing user-costs. Failing to account for an aspect of user-cost is equivalent to assuming zero cost. User costs are composed of: 

\begin{description}
	\item[Cost-to-Travel] Cost of traveling to a station or deviating from the shortest-path to utilize a station
	\item[Cost-to-Wait] Cost of waiting to charge
	\item[Cost-to-Charge] Cost of charging
	\item[Cost-of-Failure] Cost of failing to charge at a given station.
\end{description}

Every study surveyed includes cost-to-travel. For nodal demand, the cost-to-travel scales with the distance from the origin node to the station \citep{Zhu_2018_jssse, Yi_2019_e, Xiao_2020_jes, Ma_Xie_2021_trd, Kuby_2023_he, Faustino_2023_e, Gupta_2023_jes, Liu_2023_ijst, Weekx_2024_trr, Yuvaraj_2024_ia, Davatgari_2024_ejor, Tungom_2024_esa, Vijay_2024_es, Wu_2024_trd} where each origin-station pair has a unique cost. Because it is inconvenient to have to travel a substantial distance specifically to charge \citep{Rabinowitz_2023_ia}, the felt cost of traveling may increase exponentially with distance and many stations will be effectively unreachable. For travel demand, the cost-to-travel scales with the selected alternate path's distance \citep{Kuby_2005_seps, Kuby_2007_nse, Upchurch_2009_ga, Kim_2012_ijhe, MirHassani_2013_ts, Huang_Li_2015_nse, Li_Huang_2016_trc, Zhang_2017_trb, Tian_2018_s, Arslan_2019_ts, Anjos_2020_ejor, Wu_2024_trd, Zeng_2024_mt, Ala_2024_scis, Pourvaziri_2024_tre}. As overall trip length increases, the importance of deviation paths to total trip distance decreases and, often, many paths will be close enough in cost to the shortest-path to be considered viable alternatives.

The cost-to-wait is often not modeled. Optimal \gls{esn} expansion methodology is rooted in fueling network optimization where negligible at-station waiting times may be assumed. Because \glspl{bev} have shorter ranges and are slower to charge than \glspl{icev}, waiting cannot be safely assumed to be zero. Where the cost-to-wait is accounted for, a queuing model is applied. Queuing theory provides established formulae for computing an expectation of waiting time based on the distribution of arrival times, distribution of service times, and number of servers. The parameters of these formulae are set by station demand, charging speeds, and number of chargers. The relationship between arrival rate and waiting time is nonlinear and approaches infinity as arrival rate approaches the maximum throughput of the station. Queuing models are used to represent the time cost of waiting \citep{Zhu_2018_jssse, Tian_2018_s, Yi_2019_e, Wang_2022_er, Vijay_2024_es, Pourvaziri_2024_tre} or to compute the maximum allowable arrival rate or minimum number of chargers \citep{Xiao_2020_jes, Wu_2024_trd}.

The cost-to-charge is sometimes modeled and sometimes ignored. In reality, the time required to charge an \glspl{bev} is dependent on many factors including the maximum nominal rates allowed by the charger and vehicle (each of which is effected by external conditions) and the starting and final \gls{soc} of the event. Most \glspl{bev} will feature a nearly linear phase of charging followed by phase where the charge rate declines exponentially \citep{Marra_2012}. It is possible for a path featuring two charge events which are exclusively in the linear phase to be shorter than a path featuring one event which extends into the exponential decay phase. If linear charging ranges and homogeneous charging equipment are assumed then the cost-of-charging will scale linearly with path energy consumption and can be ignored. Studies which do not account for cost-to-charge  either implicitly or explicitly use this logic. Where cost to charge is accounted for, this can be done by accounting for chargers with different maximum rates and/or accounting for different final \gls{soc} \citep{Davatgari_2024_ejor, Yuvaraj_2024_ia, Zhu_2018_jssse, Upchurch_2009_ga, Zhang_2023_ijpr, Tian_2018_s, Vijay_2024_es, Wang_2022_er, Wu_2024_trd}. The price of a charge event is usually ignored but may be accounted for if the topology of the power grid is also modeled \citep{Yuvaraj_2024_ia, Gupta_2023_jes, Vijay_2024_es, Yi_2019_e}. However, a direct and causal relationship between grid topology and charging prices at any given station is not reflective of reality as charging prices usually reflect regional electricity prices \cite{Trinko_2021}.

The cost-of-failure is either an implicit or explicit element of the optimization. For a variety of reasons, a given station or path might be infeasible for a given vehicle. If no station or path is feasible for that vehicle then then the \gls{esn} has partially failed. These failures can either be priced in (maximal-coverage) or used as a constraint (set-coverage). For set-coverage formulations, the most common form requires at least a given portion of demand to be accommodated \citep{Ala_2024_scis, Anjos_2020_ejor, Arslan_2019_ts, Davatgari_2024_ejor, Pourvaziri_2024_tre, Tian_2018_s, Wu_2024_trd, Vijay_2024_es, Xiao_2020_jes, Yi_2019_e, Zhang_2023_ijpr, Zhu_2018_jssse}. A sort of hybrid approach can be taken where the optimization is formulated as a set-coverage problem but the options (stations or paths) are limited to only those which might reasonably be taken \citep{Faustino_2023_e, Gupta_2023_jes, Huang_Li_2015_nse, Kim_2012_ijhe, Kuby_2005_seps, Kuby_2007_nse, Kuby_2023_he, Li_Huang_2016_trc, Ma_Xie_2021_trd, MirHassani_2013_ts, Tungom_2024_esa, Upchurch_2009_ga, Zhang_2017_trb}.

\subsubsection{Infrastructure Costs}

Building out the optimal \gls{esn} comes at a cost. The components of infrastructure cost are:

\begin{description}
	\item[Station-Costs] Cost of all operations required to open a station before adding chargers and independent of size and power draw.
	\item[Charger-Costs] Costs of procuring and installing chargers and associated size-scaling equipment.
	\item[Grid-Costs] Cost of upgrading the grid to deliver sufficient power to the station.
\end{description}

It is common to ignore infrastructure costs altogether or to treat stations as fungible units of which a certain amount can be deployed. Where infrastructure costs are modeled, the elements of cost are modeled in a relatively common manner. Fixed-costs can be amortized among all chargers in the station and the inclusion of fixed costs means that larger stations will be more cost efficient on a per-charger basis \citep{Ala_2024_scis, Anjos_2020_ejor, Pourvaziri_2024_tre, Vijay_2024_es, Wang_2022_er, Wu_2024_trd, Xiao_2020_jes, Zhang_2023_ijpr, Zhu_2018_jssse}. Charger-costs scale linearly with the number of chargers installed \citep{Ala_2024_scis, Anjos_2020_ejor, Davatgari_2024_ejor, Gupta_2023_jes, Pourvaziri_2024_tre, Vijay_2024_es, Wang_2022_er, Wu_2024_trd, Xiao_2020_jes, Zhang_2023_ijpr, Zhu_2018_jssse} of a given type and may include operations and maintenance \citep{Yi_2019_e, Zhu_2018_jssse}. Grid-costs may be location specific or a general model and may effect the price of electricity at a station. Grid costs may scale with chargers \cite{Yi_2019_e} or may be modeled using a representation of a distribution grid \cite{Gupta_2023_jes, Vijay_2024_es}.

In general, the cost models employed provide a fixed cost for opening a station and then other costs scale with he station size and electricity demand. As fixed costs are only applied once, this incentivizes larger stations over smaller stations all else being equal. This model of costs reflects reality as larger installations come at a lower cost-per-charger price \citep{Nicholas_2019}. Power grid upgrade costs are a substantial portion of the total cost \citep{Gamage_2023} but modeling these costs is quite inexact when dealing with distribution grids \citep{Li_2024_pnas}. The decision to include or not include the various infrastructure costs reflects scoping decisions and there is a general divide between transportation focused studies and energy focused studies reflected in the absence or presence of a grid cost component.

\subsection{Approach}

Considered from a system-level perspective, the underlying goal of the optimization is multi-objective: to find a solution which maximizes level-of-service while minimizing infrastructure cost. These objectives are inherently conflicting. The literature surveyed can be separated into exact and approximate approaches. In order to compute an exact solution, one must either define an equivalence between infrastructure cost and level-of-service or optimize for one subject to a constraint on the other. The problem can then be solved using an optimal \gls{milp} or \gls{minlp} solver. Alternatively, an approximate solution may be found using dominance methodology utilizing metaheuristics such as \gls{nsga} \citep{Deb_2002_tec} or a related algorithm. Approximate approaches are used to reveal the Pareto surface between user costs and infrastructure costs and are most useful in the conceptualization phase of a project. Exact approaches solve well defined problems and will be more useful in a project's planning stage.

Regardless of approach, most papers surveyed contained continuous and integer elements. The continuous aspects of the problem concern fractional assignment of demand elements to stations and other variables such energy dispensed and queue waiting times. The integer elements are those elements which define network structure such as the location and size of stations and grid equipment. Approximate approaches tend to feature complex and multifaceted objective functions designed to give a user a complete understanding of the trade-offs in network design. Exact approaches tend to feature simple objectives and run-time optimized methodology. Exact approaches often reduce the problem to an effort to find a set of station locations which connect all demands. This simplifying assumption can be leveraged to drastically reduce the design space and improve computational efficiency \cite{Arslan_2019_ts}.

The two approaches described represent a common trade-off in optimization. Approximate methods allow for the consideration of ever more factors and complex nonlinear behavior. This comes with the drawback of being hyper-parameter dependent. In order to utilize the full capability of these methods, one needs a tremendous amount of exact data and a tremendous amount of computational cycles. Exact methods require simpler and more straightforward formulations but bring a level of transparency and can be heavily optimized for run-time scaling.

\subsection{Summary}

Optimal \gls{esn} expansion is a well studied problem. The methodology seen in the surveyed literature approaches the optimal \gls{esn} expansion problem form a variety of perspectives. These perspectives include those of transportation planners and policy makers, \gls{esn} operators, and power system operators. The studies surveyed also reflect different levels of specificity with methods designed for high-level and extremely detailed analysis. For most of these papers, advancing the methodology is the core contribution. Case studies in optimization papers are often performed on networks that are unrealistically small in scale and with complete information.

The focus of this study is to demonstrate the effects of station composition on \gls{esn} performance for long-trip travel at a strategic/policy level. As a result, the method is deliberately limited in scope and designed to be as simple as possible while retaining the capability to answer the questions asked. The specific focus of the model dictates several decisions. First, the model focuses nearly entirely on user costs. Policy makers and strategic decision makers should be primarily concerned with improving user experience and secondarily with achieving maximum efficiency in pursuit of this goal. Further, the costs of purchasing land and upgrading the distribution power grid are difficult to predict and context dependent. These factors will, no doubt, influence decisions on a micro level but will not impact a general understanding of the relationship between \gls{esn} structure of performance. As such, user costs are modeled in detail while infrastructure costs are modeled in the simplest manner possible. Second, in pursuit of a clear and transparent model, an exact approach was selected. Although metaheuristic optimization offers the ability to efficiently explore complex design spaces, it does so at the cost of hyperparameter dependence. It is impossible for one to make an obvious case for which precise form of a metaheuristic algorithm and which precise sets of tuning parameters are the best combination for the optimal \gls{esn} expansion problem. Any paper which uses such a method must devote substantial space to explaining and justifying the method itself. By contrast, an exact approach can utilize an exiting and rigorously proven optimal solver.

The method presented models all elements of user cost accounting for additional driving time, queuing time, charging time, and the cost of failing to accommodate demand and does so in an efficient manner. A key element of this method are a linearized M/M/c queuing model. This model is then applied for a case study on a very large network where the impact of \gls{esn} structure on performance at varying demand levels is demonstrated along with the very high impact of targeted expansion. The case study provides generalizable takeaways which are of value to policy makers and strategic planners. 

%The two approaches described represent a common trade-off in optimization. Approximate methods allow for the consideration of ever more factors and complex nonlinear behavior. This comes with the drawback of being hyper-parameter dependent. In order to utilize the full capability of these methods, one needs a tremendous amount of exact data and a tremendous amount of computational cycles. Exact methods require simpler and more straightforward formulations but bring a level of transparency and can be optimized.

%Problem complexity will scale exponentially with the scale of the network being optimized. Of the papers surveyed, only \cite{Arslan_2019_ts} presented a case study on a truly large network using a minimalist method.

%An \gls{esn} must be designed to meet demand. Demand may be defined by locations (nodal), or trips/tours (travel) \citep{Metais_2022}. Nodal demand formulations attempt to minimize the cost of relocating from locations where users will be when they desire to charge to charging locations. This is accomplished by assigning equipment to locations and demand nodes to equipment. Nodal demand studies have employed various formulations \citep{Kuby_2023} to optimize around technical, economic, and environmental objectives and constraints \citep{Yuvaraj_2024, Faustino_2023, Gupta_2023}. Variations introduce stochasticity \citep{Tungom_2024, Wu_Fainman_2024}, fleet operations \citep{Davatgari_2024, Ma_Xie_2021}, nonlinear queuing \citep{Liu_2023} dynamics, and over-flow relocation \citep{Weekx_2024}. Nodal demand problems are most relevant to planning on a municipal scale. Most travel is routine and short-distance \cite{NHTS_2017, NHTS_2022}, and most charging is accomplished during long-dwells \citep{Hardman_2018}. Access to low-rate charging is an important aspect of \gls{bev} user convenience \citep{Rabinowitz_2023}.
%
%Travel demand formulations attempt to minimize the added cost of charging/fueling for itineraries which are likely to exceed vehicle range. This is accomplished by assigning equipment to locations and flows to equipment. Charging and fueling add time to trips because of the time required to charge/fuel, the time required to wait for available equipment, and the time required to deviate from one's route to reach a station. Flow-capturing formulations optimize an \gls{esn} to best serve traffic flows along the routes they normally take, often the shortest path \citep{Kuby_2005, Kuby_2007, Upchurch_2009}. Flow-enabling formulations optimize and \gls{esn} to best enable traffic flows by dictating what routes they will take from a set of possible paths \citep{Kim_2012, MirHassani_2013, Huang_Li_2015, Li_Huang_2016, Zhang_2017, Arslan_2019, Anjos_2020, Wu_2024, Zeng_2024, Ala_2024}. Travel demand problems are most relevant to regional-scale planning.









